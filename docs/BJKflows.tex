%
%  untitled
%
%  Created by Andrew H. Jaffe on 2010-11-30.
%  Copyright (c) 2010 Imperial College London. All rights reserved.
%
\documentclass[]{article}

% Use utf-8 encoding for foreign characters
\usepackage[utf8]{inputenc}

% Setup for fullpage use
\usepackage{fullpage}

% Uncomment some of the following if you use the features
%
% Running Headers and footers
%\usepackage{fancyhdr}

% Multipart figures
%\usepackage{subfigure}

% More symbols
%\usepackage{amsmath}
%\usepackage{amssymb}
%\usepackage{latexsym}

% Surround parts of graphics with box
\usepackage{boxedminipage}

% Package for including code in the document
\usepackage{listings}

% If you want to generate a toc for each chapter (use with book)
\usepackage{minitoc}

% This is now the recommended way for checking for PDFLaTeX:
\usepackage{ifpdf}

%\newif\ifpdf
%\ifx\pdfoutput\undefined
%\pdffalse % we are not running PDFLaTeX
%\else
%\pdfoutput=1 % we are running PDFLaTeX
%\pdftrue
%\fi

\ifpdf
\usepackage[pdftex]{graphicx}
\else
\usepackage{graphicx}
\fi
\title{Using BJK for flows}
\author{  }

\date{2010-11-30}

\begin{document}

\ifpdf
\DeclareGraphicsExtensions{.pdf, .jpg, .tif}
\else
\DeclareGraphicsExtensions{.eps, .jpg}
\fi

\maketitle

% 
% \begin{abstract}
% \end{abstract}
% 
% \section{Introduction}

   OK, I would use the method to do something similar to what you did in the last paper to calculate $P(k)$ in bands. If you have the code to do the correlation matrix, that's basically all you need.  If $s_i$ are the LOS velocities, we want to write
\begin{equation}
    C_{ij}[P(k)] = \left<s_i s_j\right> = \sum_b  P_b S_{ij,b} + N_{ij}
\end{equation}
where $P(k)$ is top-hat in band b (or any other shape you want) with bandpower $P_b$, so the $S_{ij,b}$ are the partial derivatives with respect to the $P_b$. $N_{ij} = \sigma^2_i \delta_{ij}$ is the noise covariance

For this case, the partials are given by the expressions from the code and paper:
\begin{equation}
    \left<s_i s_j\right> = \int \frac{4\pi k^2\;dk}{(2\pi)^3} P_v(k)f_{ij}(k)
\end{equation}
with
\begin{equation}
    f_{ij}(k) = 
    \frac{(r_i - \mu_{ij} r_j)(r_j-\mu_{ij} r_i)}{r_i^2 + r_j^2-2\mu_{ij} r_i r_j}
       \left[f_\perp(kr)-f_\parallel(kr)\right]
       +\mu_{ij} f_\perp(kr)
\end{equation}
where ${\hat r_i}\cdot{\hat r}_j=\mu_{ij}$ is the cosine of the angle between the two objects and the geometrical functions, related to spherical bessel functions, are
\begin{equation}
    f_\perp(x)= (\sin x - x \cos x)/x^3 = j_1(x)/x
\end{equation}
and 
\begin{equation}
    f_\perp(x) - f_\parallel(x) = j_2(x) = \left[(3-x^2)\sin x - 3x\cos x\right]/x^2
\end{equation}

With $N$ and the $S_{,b}$ we can apply the BJK formalism directly. From some set of previous $P_b$, we calculate the step to the next iteration as
\begin{equation}
    \delta P_b = \sum_{b'}\frac12F^{-1}_{bb'}\mathrm{Tr}\left[(ss^T-C)(C^{-1}S_{,b'}C^{-1})\right]
\end{equation}
and the Fisher matrix is
\begin{equation}
    F_{bb}= \frac12\textrm{Tr}\left[C^{-1}S_{,b}C^{-1}S_{,b'}\right]
\end{equation}
where all matrix operations inside the square brackets, including the traces, are done on the (suppressed) $ij$ object indices (and so $ss^T$ is just $s_i s_j$)

\bibliographystyle{plain}
\bibliography{}
\end{document}
